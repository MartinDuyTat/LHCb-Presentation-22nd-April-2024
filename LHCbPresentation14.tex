%%%%%%%%%%%%%%%%%%%%%%%%%%%%%%%%%%%%%%%%%%%%%%%%%%%%%%%%%%%%%%%%%%%%%%
% Overleaf (WriteLaTeX) Example: Molecular Chemistry Presentation
%
% Source: http://www.overleaf.com
%
% In these slides we show how Overleaf can be used with standard 
% chemistry packages to easily create professional presentations.
% 
% Feel free to distribute this example, but please keep the referral
% to overleaf.com
% 
%%%%%%%%%%%%%%%%%%%%%%%%%%%%%%%%%%%%%%%%%%%%%%%%%%%%%%%%%%%%%%%%%%%%%%

\documentclass{beamer}

\mode<presentation>
{
  \usetheme{Madrid}       % or try default, Darmstadt, Warsaw, ...
  \usecolortheme{default} % or try albatross, beaver, crane, ...
  \usefonttheme{default}    % or try default, structurebold, ...
  \setbeamertemplate{navigation symbols}{}
  \setbeamertemplate{caption}[numbered]
} 

\usepackage[english]{babel}
\usepackage[utf8x]{inputenc}
\usepackage{graphicx}
\usepackage{hyperref}
  \hypersetup{colorlinks=true}
  \hypersetup{urlcolor=blue}
  \hypersetup{linkcolor = .}
\usepackage{xcolor}
\usepackage{siunitx}
  \sisetup{separate-uncertainty = true}
\usepackage{physics}
\usepackage[font=small,labelfont=bf,justification=centering]{caption}
\usepackage{subcaption}
\usepackage[en-GB]{datetime2}
\usepackage{overpic}
\usepackage{feynmp}
\DeclareGraphicsRule{*}{mps}{*}{}
\usepackage{scalerel}
\newcommand{\mylbrace}[2]{\vspace{#2pt}\hspace{6pt}\scaleleftright[\dimexpr5pt+#1\dimexpr0.06pt]{\lbrace}{\rule[\dimexpr2pt-#1\dimexpr0.5pt]{-4pt}{#1pt}}{.}}
\newcommand{\myrbrace}[2]{\vspace{#2pt}\scaleleftright[\dimexpr5pt+#1\dimexpr0.06pt]{.}{\rule[\dimexpr2pt-#1\dimexpr0.5pt]{-4pt}{#1pt}}{\rbrace}\hspace{6pt}}
\usepackage{ulem} % Line across text
\newcommand{\white}[1]{{\textcolor{white}{#1}}} % White text

% Here's where the presentation starts, with the info for the title slide
\title[$h^+h^-\pi^+\pi^-$]{Model-independent measurement of $\gamma$ in $B^\pm\to[h^+h^-\pi^+\pi^-]_Dh^\pm$ at LHCb and BESIII}

\author{Martin Tat}
\institute{Oxford LHCb}
\date{22nd April 2024}

\titlegraphic{\includegraphics[height = 2cm]{lhcb.jpg}\hspace{1cm}~%
              \includegraphics[height = 2cm]{OxfordLogo.pdf}\hspace{1cm}~%
              \includegraphics[height = 2cm]{bes3.jpg}}

\begin{document}

\begin{frame}
  \titlepage
\end{frame}

% These three lines create an automatically generated table of contents.
%\begin{frame}{Outline}
%  \tableofcontents
%\end{frame}

\section{Introduction}

\begin{frame}{Introduction}
  \begin{center}
    \large{Last presentation since starting my PhD journey 1295 days ago!}
  \end{center}
  \vspace{0.1cm}
  \begin{enumerate}
    \setlength\itemsep{0.7em}
    \item{October 2020: Sensitivity studies with $B^\pm\to [K^+K^-\pi^+\pi^-]_DK^\pm$}
    \item{April 2021: First data fit reveals large tension in $\gamma$}
  \end{enumerate}
  \begin{figure}
    \centering
    \includegraphics[width = 0.45\textwidth]{Plots/GammaTensionMeme.jpg}
  \end{figure}
  \begin{enumerate}
    \setlength\itemsep{0.7em}
    \setcounter{enumi}{3}
    \item{January 2023: Model-dependent $\gamma$ publication with $3~\sigma$ tension}
    \item{March 2024: BESIII $c_i$/$s_i$ analysis approved by review committee}
    \item{April 2024: Presented $B^\pm\to [h^+h^-\pi^+\pi^-]_Dh^\pm$ to B2OC}
  \end{enumerate}
\end{frame}

\begin{frame}{Recap of BESIII $D^0\to K^+K^-\pi^+\pi^-$ strong-phase results}
  \begin{center}
    With additional BESIII data ($\SI{16}{\per\femto\barn}$), $c_i$/$s_i$ agree perfectly with model \\
    Analysis approved by review committee, paper currently in review
  \end{center}
  \vspace{-0.3cm}
  \begin{columns}
    \begin{column}{0.5\textwidth}
      \vspace{-0.5cm}
      \begin{align*}
        c_1 =& -0.28 \pm 0.09 \pm 0.01 \\
        s_1 =& -0.68 \pm 0.24 \pm 0.04 \\
        c_2 =& +0.83 \pm 0.04 \pm 0.01 \\
        s_2 =& -0.18 \pm 0.19 \pm 0.03 \\
        c_3 =& +0.83 \pm 0.03 \pm 0.01 \\
        s_3 =& +0.27 \pm 0.17 \pm 0.03 \\
        c_4 =& -0.28 \pm 0.10 \pm 0.01 \\
        s_4 =& +0.54 \pm 0.28 \pm 0.04
      \end{align*}
    \end{column}
    \begin{column}{0.5\textwidth}
      \begin{figure}
        \centering
        \includegraphics[width=0.9\textwidth]{Plots/cisi_FitResults_Model.pdf}
      \end{figure}
    \end{column}
  \end{columns}
  \begin{center}
    Measured values (black) are consistent and close to LHCb model predictions (blue), so central value of $\gamma$ is not expected to change much
  \end{center}
\end{frame}

\begin{frame}{BESIII preliminary $D^0\to\pi^+\pi^-\pi^+\pi^-$ strong-phase results}
  \vspace{0.3cm}
  \begin{itemize}
    \setlength\itemsep{1.0em}
    \item{Binned strong-phase analysis of $D^0\to\pi^+\pi^-\pi^+\pi^-$ uses the $2\times5$ ``optimal'' binning scheme with $3$~fb$^{-1}$ $\psi(3770)$}
    \item{Earlier CLEO-c analysis with $0.8$~fb$^{-1}$ \href{https://link.springer.com/article/10.1007/JHEP01(2018)144}{JHEP \textbf{01} (2018) 144}}
    \item{New BESIII analysis uses a new binning scheme optimised with a BESIII amplitude model \href{https://arxiv.org/abs/2312.02524}{arXiv:2312.02524}}
    \begin{itemize}
      \item{Amplitude model constructed from a larger data set}
      \item{In principle more sensitive}
    \end{itemize}
    \item{Two binning schemes are available:}
    \begin{itemize}
      \item{We use the more sensitive ``optimal'' binning with $Q = 0.85$}
      \item{The other ``equal $\delta$'' binning has $Q = 0.80$}
    \end{itemize}
    \item{Analysis also approved by review committee, currently in paper review}
  \end{itemize}
\end{frame}

\begin{frame}{BESIII preliminary $D^0\to\pi^+\pi^-\pi^+\pi^-$ strong-phase results}
  \begin{center}
    Small differences between model prediction and measurement, but data points are generally close to the unit circle
  \end{center}
  \vspace{-0.3cm}
  \begin{columns}
    \begin{column}{0.50\textwidth}
      \vspace{-0.5cm}
      \begin{align*}
        c_1 =& +0.12 \pm 0.09 \pm 0.02 \\
        s_1 =& -0.42 \pm 0.21 \pm 0.04 \\
        c_2 =& +0.74 \pm 0.04 \pm 0.02 \\
        s_2 =& -0.39 \pm 0.16 \pm 0.06 \\
        s_3 =& -0.25 \pm 0.12 \pm 0.03 \\
        c_3 =& +0.81 \pm 0.03 \pm 0.01 \\
        c_4 =& +0.42 \pm 0.06 \pm 0.02 \\
        s_4 =& +0.86 \pm 0.19 \pm 0.07 \\
        c_5 =& -0.27 \pm 0.09 \pm 0.03 \\
        s_5 =& -0.22 \pm 0.25 \pm 0.08
      \end{align*}
    \end{column}
    \begin{column}{0.50\textwidth}
      \begin{figure}
        \centering
        \includegraphics[width=1.0\textwidth]{Plots/CiSiOptim.pdf}
      \end{figure}
    \end{column}
  \end{columns}
  \begin{center}
    Plan: Publish measurement of $\gamma$ using both $K^+K^-\pi^+\pi^-$ and $\pi^+\pi^-\pi^+\pi^-$
  \end{center}
\end{frame}

\section{Introduction to model-independent binned measurement of \texorpdfstring{$\gamma$}{gamma}}
\begin{frame}{The BPGGSZ method}
  \begin{center}
    \begin{minipage}{0.6\textwidth}
      \begin{block}{Event yield in bin $i$}
        \footnotesize
        $N^-_i = h_{B^-}\big(F_i + (x_-^2 + y_-^2)\bar{F_i} + 2\sqrt{F_i\bar{F_i}}(x_-c_i + y_-s_i)\big)$ \\
        $N^+_{-i} = h_{B^+}\big(F_i + (x_+^2 + y_+^2)\bar{F_i} + 2\sqrt{F_i\bar{F_i}}(x_+c_i + y_+s_i)\big)$
      \end{block}
    \end{minipage}
  \end{center}
  \begin{itemize}
    \item{CP observables:}
    \begin{itemize}
      \item{$x_\pm^{DK} = r_B^{DK}\cos(\delta_B^{DK}\pm\gamma)$, \quad $y_\pm^{DK} = r_B^{DK}\sin(\delta_B^{DK}\pm\gamma)$}
      \item{$x_\xi^{D\pi} = \Re(\xi^{D\pi})$, $y_\xi^{D\pi} = \Im(\xi^{D\pi})$ $\quad\quad\Big(\xi^{D\pi} = \frac{r_B^{D\pi}}{r_B^{DK}}e^{i(\delta_B^{D\pi} - \delta_B^{DK})}\Big)$}
    \end{itemize}
    \item{Fractional bin yield:}
    \begin{itemize}
      \item{$F_i = \frac{\int_i\dd{\Phi}|\mathcal{A}(D^0)|^2}{\sum_j\int_j\dd{\Phi}\abs{\mathcal{A}(D^0)}^2}$}
      \item{Floated in the fit, mostly constrained by $B^\pm\to D\pi^\pm$}
    \end{itemize}
  \end{itemize}
  \begin{itemize}
    \item{Amplitude averaged strong phases:}
    \begin{center}
      $c_i = \frac{\int_i\dd{\Phi}|\mathcal{A}(D^0)||\mathcal{A}(\bar{D^0})|\cos(\delta_D)}{\sqrt{\int_i\dd{\Phi}\abs{\mathcal{A}(D^0)}^2\int_i\dd{\Phi}\abs{\mathcal{A}(\bar{D^0})}^2}}$ \quad $s_i = \frac{\int_i\dd{\Phi}|\mathcal{A}(D^0)||\mathcal{A}(\bar{D^0})|\sin(\delta_D)}{\sqrt{\int_i\dd{\Phi}\abs{\mathcal{A}(D^0)}^2\int_i\dd{\Phi}\abs{\mathcal{A}(\bar{D^0})}^2}}$
    \end{center}
  \end{itemize}
\end{frame}

\begin{frame}{Model-dependent measurement with $D\to K^+K^-\pi^+\pi^-$}
  \begin{center}
    \large From the phase-space binned asymmetries, we obtain:\\
    \vspace{0.2cm}
    $\gamma = (116^{+12}_{-14})^\circ$
  \end{center}
  \vspace{-0.2cm}
  \begin{figure}[htb]
    \centering
    \begin{subfigure}{0.5\textwidth}
      \includegraphics[width=1\textwidth]{Plots/BinAsymmetries_dk_ModelDependent.pdf}
    \end{subfigure}%
    \begin{subfigure}{0.5\textwidth}
      \includegraphics[width=1\textwidth]{Plots/gammacharm_lhcb_KKpipi_GLW_KKpipi_GGSZ_lhcb_2020_beauty_and_charm_g_d_dk_ModelDependent.pdf}
    \end{subfigure}
    \vspace{-0.5cm}
    \caption*{\tiny\href{https://link.springer.com/article/10.1140/epjc/s10052-023-11560-5}{Eur. Phys. J. C \textbf{83}, 547 (2023)}}
  \end{figure}
  \vspace{-0.5cm}
  \begin{center}
    {\large How will this evolve with model-independent BESIII inputs? Will the $3\sigma$ tension reduce?}
  \end{center}
\end{frame}

\section{Global fit}
\begin{frame}{Global fit}
  \begin{center}
    {\large Global fit of $K^+K^-\pi^+\pi^-$ remains as in model-dependent publication:}
  \end{center}
  \begin{figure}
    \centering
    \includegraphics[width = 0.9\textwidth,trim={0 0 0 0},clip=true]{Plots/d2kkpipi_fiveL_allDP.pdf}
  \end{figure}
  \vspace{-0.5cm}
  \begin{itemize}
    \item{$B^\pm\to[K^+K^-\pi^+\pi^-]_Dh^\pm$ signal yield:}
    \begin{itemize}
      \item{$B^\pm\to DK^\pm$: $3051 \pm 38$}
      \item{$B^\pm\to D\pi^\pm$: $44356 \pm 218$}
    \end{itemize}
  \end{itemize}
\end{frame}

\begin{frame}{Global fit}
  \begin{center}
    {\large How do we include the $\pi^+\pi^-\pi^+\pi^-$ mode?}
  \end{center}
  \begin{itemize}
    \setlength\itemsep{1.5em}
    \item{We \underline{have already studied} $B^\pm\to[\pi^+\pi^-\pi^+\pi^-]_Dh^\pm$ for phase-space integrated measurement}
    \begin{enumerate}
      \setlength\itemsep{0.5em}
      \item{Different $D$ daughter PID cuts in stripping}
      \item{No $D\to K\pi\pi\pi\pi^0$ background}
      \item{Charmless background recalculated using the sideband}
      \item{Use same BDT}
      \item{No additional peaking backgrounds}
    \end{enumerate}
    \item{Sort candidates into phase-space bins using BESIII binning scheme}
    \item{Can fit separately or simultaneously with $K^+K^-\pi^+\pi^-$}
  \end{itemize}
\end{frame}

\begin{frame}{Global fit}
  \begin{center}
    {\large Global fit of $\pi^+\pi^-\pi^+\pi^-$ has a good fit quality:}
  \end{center}
  \begin{figure}
    \centering
    \includegraphics[width = 0.9\textwidth,trim={0 0 0 0},clip=true]{Plots/d2pipipipi_fiveL_allDP.pdf}
  \end{figure}
  \vspace{-0.5cm}
  \begin{itemize}
    \item{$B^\pm\to[\pi^+\pi^-\pi^+\pi^-]_Dh^\pm$ signal yield:}
    \begin{itemize}
      \item{$B^\pm\to DK^\pm$: $8745 \pm 105$}
      \item{$B^\pm\to D\pi^\pm$: $126314 \pm 385$}
    \end{itemize}
  \end{itemize}
\end{frame}

\section{CP fit}
\begin{frame}{CP fit}
  \begin{center}
    {\large After global fit, perform a ``CP fit'' to study CP violation:}
  \end{center}
  \begin{itemize}
    \setlength\itemsep{1.0em}
    \item{Split candidates by:}
    \begin{enumerate}
      \item{$B^+$ and $B^-$ charges}
      \item{$B^\pm\to DK^\pm$ and $B^\pm\to D\pi^\pm$ decays}
      \item{$D$ phase-space bins}
    \end{enumerate}
    \item{Combinatorial and low-mass backgrounds are floating in each category}
    \item{Parameterise signal yields in terms of $x_\pm^{DK}$, $y_\pm^{DK}$, $x_\xi^{D\pi}$, $y_\xi^{D\pi}$}
    \item{$2N - 1$ floating $F_i$ parameters}
    \item{\underline{$c_i$ and $s_i$ are Gaussian constrained}}
  \end{itemize}
\end{frame}

\begin{frame}{CP fit categories}
  \begin{center}
    {\large Summary of free parameters in the CP fit:}
  \end{center}
  \vspace{-0.5cm}
  \begin{columns}
    \begin{column}{0.5\textwidth}
      \begin{center}
        $K^+K^-\pi^+\pi^-$ \\
        $2\times2\times2\times4 = 32$ categories
      \end{center}
      \begin{itemize}
        \setlength\itemsep{0.0em}
        \item{6 CP observables}
        \item{7 $F_i$ parameters}
        \item{8 $c_i$ and $s_i$ parameters}
        \item{32 combinatorial yields}
        \item{32 low mass yields}
        \item{4 global normalisations}
        \item{Total: 89 parameters}
      \end{itemize}
    \end{column}
    \begin{column}{0.5\textwidth}
      \begin{center}
        $\pi^+\pi^-\pi^+\pi^-$ \\
        $2\times2\times2\times5 = 40$ categories
      \end{center}
      \begin{itemize}
        \setlength\itemsep{0.0em}
        \item{6 CP observables}
        \item{9 $F_i$ parameters}
        \item{10 $c_i$ and $s_i$ parameters}
        \item{40 combinatorial yields}
        \item{40 low mass yields}
        \item{4 global normalisations}
        \item{Total: 109 parameters}
      \end{itemize}
    \end{column}
  \end{columns}
  \vspace{0.3cm}
  \begin{center}
    In a combined fit where CP observables are shared, there are $89 + 109 - 6 = 192$ parameters
  \end{center}
\end{frame}

\begin{frame}{CP fit bin asymmetry}
  \begin{center}
    {\large Example of bin asymmetry in $D\to K^+K^-\pi^+\pi^-$ bin $-3$:}
  \end{center}
  \begin{figure}
    \centering
    \includegraphics[width = 0.9\textwidth,trim={0 10cm 0 0},clip=true]{Plots/d2kkpipi_fiveL_binm3.pdf}
  \end{figure}
\end{frame}

\begin{frame}{CP fit bin asymmetry}
  \begin{center}
    {\large Example of bin asymmetry in $D\to\pi^+\pi^-\pi^+\pi^-$ bin $+5$:}
  \end{center}
  \begin{figure}
    \centering
    \includegraphics[width = 0.9\textwidth,trim={0 10cm 0 0},clip=true]{Plots/d2pipipipi_fiveL_binp5.pdf}
  \end{figure}
\end{frame}

\begin{frame}{Bin asymmetries}
  \begin{center}
    $B^\pm\to[K^+K^-\pi^+\pi^-]_Dh^\pm$ bin asymmetries
  \end{center}
  \begin{figure}
    \centering
    \begin{subfigure}{0.5\textwidth}
      \centering
      \includegraphics[width=1.0\textwidth]{Plots/BinAsymmetries_dk_KKpipi.pdf}
      \caption*{$B^\pm\to DK^\pm$}
    \end{subfigure}%
    \begin{subfigure}{0.5\textwidth}
      \centering
      \includegraphics[width=1.0\textwidth]{Plots/BinAsymmetries_dpi_KKpipi.pdf}
      \caption*{$B^\pm\to D\pi^\pm$}
    \end{subfigure}
  \end{figure}
\end{frame}

\begin{frame}{Bin asymmetries}
  \begin{center}
    $B^\pm\to[\pi^+\pi^-\pi^+\pi^-]_Dh^\pm$ bin asymmetries
  \end{center}
  \begin{figure}
    \centering
    \begin{subfigure}{0.5\textwidth}
      \centering
      \includegraphics[width=1.0\textwidth]{Plots/BinAsymmetries_dk_pipipipi.pdf}
      \caption*{$B^\pm\to DK^\pm$}
    \end{subfigure}%
    \begin{subfigure}{0.5\textwidth}
      \centering
      \includegraphics[width=1.0\textwidth]{Plots/BinAsymmetries_dpi_pipipipi.pdf}
      \caption*{$B^\pm\to D\pi^\pm$}
    \end{subfigure}
  \end{figure}
\end{frame}

\begin{frame}{Strong-phase parameters in CP fit}
  \begin{center}
    {\large Why are $c_i$ and $s_i$ Gaussian constrained?}
  \end{center}
  \begin{itemize}
    \setlength\itemsep{1.0em}
    \item{Previous BPGGSZ analyses have kept $c_i$ and $s_i$ fixed}
    \begin{enumerate}
      \item{$c_i$ and $s_i$ uncertainties are added as a systematic through smearing}
      \item{Convenient for calculating correlations between different analyses}
      \item{Appropriate when $c_i$ and $s_i$ uncertainties are \underline{small}}
    \end{enumerate}
    \item{In four-body analyses, uncertainties on $\gamma$ from $c_i$ and $s_i$ are almost the same size as the statistical uncertainty}
    \item{Large $s_i$ uncertainties introduces non-Gaussian uncertainties on $y_\pm$}
    \item{$\gamma$ moves significantly when fixing $s_i$ instead of constraining them}
    \item{These effects are largest for $K^+K^-\pi^+\pi^-$, but are also seen in $\pi^+\pi^-\pi^+\pi^-$ and in the combined fit}
  \end{itemize}
\end{frame}

\begin{frame}{Likelihood scan of CP observables}
  \begin{center}
    $x_\pm^{DK}$ agree well between likelihood scan and Hesse approximation
  \end{center}
  \begin{figure}
    \centering
    \begin{subfigure}{0.5\textwidth}
      \centering
      \includegraphics[width=1.0\textwidth]{Plots/A_xm_dk_likelihood_scan_KKpipi.pdf}
      \vspace{-0.3cm}
      \caption*{$x_-^{DK}$}
    \end{subfigure}%
    \begin{subfigure}{0.5\textwidth}
      \centering
      \includegraphics[width=1.0\textwidth]{Plots/A_xp_dk_likelihood_scan_KKpipi.pdf}
      \vspace{-0.3cm}
      \caption*{$x_+^{DK}$}
    \end{subfigure}
    \caption*{$D^0\to K^+K^-\pi^+\pi^-$}
  \end{figure}
\end{frame}

\begin{frame}{Likelihood scan of CP observables}
  \begin{center}
    $x_\pm^{DK}$ agree well between likelihood scan and Hesse approximation
  \end{center}
  \begin{figure}
    \centering
    \begin{subfigure}{0.5\textwidth}
      \centering
      \includegraphics[width=1.0\textwidth]{Plots/A_xm_dk_likelihood_scan_pipipipi.pdf}
      \vspace{-0.3cm}
      \caption*{$x_-^{DK}$}
    \end{subfigure}%
    \begin{subfigure}{0.5\textwidth}
      \centering
      \includegraphics[width=1.0\textwidth]{Plots/A_xp_dk_likelihood_scan_pipipipi.pdf}
      \vspace{-0.3cm}
      \caption*{$x_+^{DK}$}
    \end{subfigure}
    \caption*{$D^0\to\pi^+\pi^-\pi^+\pi^-$}
  \end{figure}
\end{frame}

\begin{frame}{Likelihood scan of CP observables}
  \begin{center}
    $y_\pm^{DK}$ diverges from Hesse approximation outside $1\sigma$
  \end{center}
  \begin{figure}
    \centering
    \begin{subfigure}{0.5\textwidth}
      \centering
      \includegraphics[width=1.0\textwidth]{Plots/A_ym_dk_likelihood_scan_KKpipi.pdf}
      \vspace{-0.3cm}
      \caption*{$y_-^{DK}$}
    \end{subfigure}%
    \begin{subfigure}{0.5\textwidth}
      \centering
      \includegraphics[width=1.0\textwidth]{Plots/A_yp_dk_likelihood_scan_KKpipi.pdf}
      \vspace{-0.3cm}
      \caption*{$y_+^{DK}$}
    \end{subfigure}
    \caption*{$D^0\to K^+K^-\pi^+\pi^-$}
  \end{figure}
\end{frame}

\begin{frame}{Likelihood scan of CP observables}
  \begin{center}
    $y_\pm^{DK}$ diverges from Hesse approximation outside $1\sigma$
  \end{center}
  \begin{figure}
    \centering
    \begin{subfigure}{0.5\textwidth}
      \centering
      \includegraphics[width=1.0\textwidth]{Plots/A_ym_dk_likelihood_scan_pipipipi.pdf}
      \vspace{-0.3cm}
      \caption*{$y_-^{DK}$}
    \end{subfigure}%
    \begin{subfigure}{0.5\textwidth}
      \centering
      \includegraphics[width=1.0\textwidth]{Plots/A_yp_dk_likelihood_scan_pipipipi.pdf}
      \vspace{-0.3cm}
      \caption*{$y_+^{DK}$}
    \end{subfigure}
    \caption*{$D^0\to\pi^+\pi^-\pi^+\pi^-$}
  \end{figure}
\end{frame}

\begin{frame}{Likelihood scan of CP observables}
  \begin{center}
    {\large What do the likelihood scans tell us?}
  \end{center}
  \begin{itemize}
    \setlength\itemsep{1.5em}
    \item{Uncertainties from $c_i$ and $s_i$ are significant, which justifies Gaussian constraining $c_i$ and $s_i$}
    \item{Non-Gaussian uncertainties means GammaCombo cannot be used}
    \item{New strategy:}
    \begin{enumerate}
      \setlength\itemsep{0.5em}
      \item{Produce a likelihood function from CP fit}
      \item{Interpret CP observables in terms of $\gamma$, etc}
      \item{Must \underline{profile} all nuisance parameters ($F_i$, $c_i$, $s_i$, backgrounds yields, normalisation constants)}
      \item{Provide direct measurements of $\gamma$, $\delta_B$ and $r_B$ without GammaCombo}
    \end{enumerate}
  \end{itemize}
\end{frame}

\section{Systematic uncertainties}
\begin{frame}{Summary of LHCb internal systematic uncertainties}
  \begin{center}
    Internal LHCb systematic uncertainties from model-dependent analysis of $K^+K^-\pi^+\pi^-$:
  \end{center}
  \vspace{-0.3cm}
  \scriptsize
  \vspace{0.02cm}
  \begin{center}
    \begin{tabular}{lcccccc}
        \hline
        Source & $x_-^{DK}$ & $y_-^{DK}$ & $x_+^{DK}$ & $y_+^{DK}$ & $x_\xi^{D\pi}$ & $y_\xi^{D\pi}$ \\
        \hline
        Statistical                                                & $2.87$ & $3.40$ & $2.51$ & $3.05$ & $4.24$ & $5.17$ \\
        \hline
        Mass shape                                                 & $0.02$ & $0.02$ & $0.03$ & $0.06$ & $0.02$ & $0.04$ \\
        Bin-dependent mass shape                                   & $0.11$ & $0.05$ & $0.10$ & $0.19$ & $0.68$ & $0.16$ \\
        PID efficiency                                             & $0.02$ & $0.02$ & $0.03$ & $0.06$ & $0.02$ & $0.04$ \\
        Low-mass background model                                  & $0.02$ & $0.02$ & $0.03$ & $0.04$ & $0.02$ & $0.02$ \\
        Charmless background                                       & $0.14$ & $0.15$ & $0.12$ & $0.14$ & $0.01$ & $0.02$ \\
        $C\!P$ violation in low-mass background                    & $0.01$ & $0.10$ & $0.08$ & $0.12$ & $0.07$ & $0.26$ \\
        Semi-leptonic $b$-hadron decays                            & $0.05$ & $0.27$ & $0.06$ & $0.01$ & $0.07$ & $0.19$ \\
        Semi-leptonic charm decays                                 & $0.02$ & $0.07$ & $0.03$ & $0.15$ & $0.06$ & $0.24$ \\
        $D\to K^-\pi^+\pi^-\pi^+$ background                       & $0.11$ & $0.05$ & $0.07$ & $0.04$ & $0.09$ & $0.05$ \\
        $\Lambda_b\to pD\pi^-$ background                          & $0.01$ & $0.25$ & $0.14$ & $0.04$ & $0.06$ & $0.34$ \\
        $D\to K^-\pi^+\pi^-\pi^+\pi^0$ background                  & $0.30$ & $0.05$ & $0.19$ & $0.07$ & $0.05$ & $0.01$ \\
        Fit bias                                                   & $0.06$ & $0.05$ & $0.13$ & $0.02$ & $0.06$ & $0.13$ \\
        \hline
        Total LHCb systematic                                      & $0.37$ & $0.43$ & $0.34$ & $0.32$ & $0.70$ & $0.57$ \\
        \hline
    \end{tabular}
  \end{center}
  \begin{center}
    {\normalsize Give systematic uncertainties in terms of CP observables (not $\gamma$) since these are more Gaussian and better behaved}
  \end{center}
\end{frame}

\begin{frame}{Summary of LHCb internal systematic uncertainties}
  \begin{center}
    Internal LHCb systematics for $\pi^+\pi^-\pi^+\pi^-$ have not been calculated yet, but the plan is to run the same procedure from $K^+K^-\pi^+\pi^-$ on this mode during review, with minor simplifications:
  \end{center}
  \begin{enumerate}
    \setlength\itemsep{1.5em}
    \item{No $K\pi\pi\pi$ background}
    \item{No $K\pi\pi\pi\pi^0$ background}
    \item{No $D$ semileptonic background}
  \end{enumerate}
  \vspace{0.2cm}
  \begin{center}
    Reminder: LHCb internal systematic uncertainties are expected to be an order of magnitude smaller than statistical uncertainties!
  \end{center}
\end{frame}

\section{Interpretation}
\begin{frame}{Interpretation strategy}
  \begin{center}
    {\large From CP fit, we have a (negative log) likelihood function with nuisance parameters $n_k$:}
  \end{center}
  \begin{equation*}
    \mathcal{L}(x_-^{DK}, y_-^{DK}, x_+^{DK}, y_+^{DK}, x_\xi^{D\pi}, y_\xi^{D\pi}, \{n_k\})
  \end{equation*}
  \vspace{0.1cm}
  \begin{center}
    {\large Express in terms of physics parameters:}
  \end{center}
  \begin{equation*}
    \mathcal{L}(\gamma, \delta_B^{DK}, r_B^{DK}, \delta_B^{D\pi}, r_B^{D\pi}, \{n_k\})
  \end{equation*}
  \vspace{0.1cm}
  \begin{center}
    {\normalsize In this step, also add a Gaussian smearing term on CP observables to account for internal LHCb systematics}
  \end{center}
\end{frame}

\begin{frame}{Interpretation toys}
  \begin{center}
    We can perform toy studies on the interpretation fit, but we do \underline{not} expect these to behave very Gaussian...
  \end{center}
  \begin{figure}
    \centering
    \begin{subfigure}{0.5\textwidth}
      \centering
      \includegraphics[width=1.0\textwidth]{Plots/gamma_pull_toys_KKpipi.pdf}
      \vspace{-0.3cm}
      \caption*{$K^+K^-\pi^+\pi^-$}
    \end{subfigure}%
    \begin{subfigure}{0.5\textwidth}
      \centering
      \includegraphics[width=1.0\textwidth]{Plots/gamma_pull_toys_pipipipi.pdf}
      \vspace{-0.3cm}
      \caption*{$\pi^+\pi^-\pi^+\pi^-$}
    \end{subfigure}
    \vspace{-0.5cm}
    \caption*{$\gamma$ pull distributions}
  \end{figure}
  \vspace{-0.3cm}
  \begin{center}
    Indeed, small but significant biases are observed!\\
    Use pull distributions to correct central values of physics parameters
  \end{center}
\end{frame}

\begin{frame}{Interpretation toys}
  \begin{center}
    We can perform toy studies on the interpretation fit, but we do \underline{not} expect these to behave very Gaussian...
  \end{center}
  \begin{figure}
    \centering
    \begin{subfigure}{0.5\textwidth}
      \centering
      \includegraphics[width=1.0\textwidth]{Plots/gamma_pull_toys_KKpipi.pdf}
      \vspace{-0.3cm}
      \caption*{$K^+K^-\pi^+\pi^-$}
    \end{subfigure}%
    \begin{subfigure}{0.5\textwidth}
      \centering
      \includegraphics[width=1.0\textwidth]{Plots/gamma_pull_toys_pipipipi.pdf}
      \vspace{-0.3cm}
      \caption*{$\pi^+\pi^-\pi^+\pi^-$}
    \end{subfigure}
    \vspace{-0.5cm}
    \caption*{$\gamma$ pull distributions}
  \end{figure}
  \vspace{-0.3cm}
  \begin{center}
    The absolute bias corrections are:\\
    $K^+K^-\pi^+\pi^-$: $+5.6^\circ$, $\pi^+\pi^-\pi^+\pi-$: $-3.0^\circ$, combined: $-3.0^\circ$
  \end{center}
\end{frame}

\begin{frame}{Interpretation results}
  \begin{center}
    {\large Results from interpretation of $K^+K^-\pi^+\pi^-$, after correcting for biases in central values (not uncertainties):}
  \end{center}
  \vspace{-0.5cm}
  \begin{columns}
    \begin{column}{0.5\textwidth}
      \begin{center}
        Model independent
      \end{center}
      \begin{align*}
        \gamma =& (117 \pm 15)^\circ \\
        \delta_B^{DK} =& (83 \pm 12)^\circ \\
        r_B^{DK} =& (12.1 \pm 2.6)\times10^{-2} \\
        \delta_B^{D\pi} =& (295 \pm 74)^\circ \\
        r_B^{D\pi} =& (0 \pm 5)\times10^{-3}
      \end{align*}
    \end{column}
    \begin{column}{0.5\textwidth}
      \begin{center}
        Model dependent
      \end{center}
      \begin{align*}
        \gamma =& (116^{+12}_{-14})^\circ \\
        \delta_B^{DK} =& (81^{+14}_{-13})^\circ \\
        r_B^{DK} =& (11.0 \pm 2.0)\times10^{-2} \\
        \delta_B^{D\pi} =& (298^{+62}_{-118})^\circ \\
        r_B^{D\pi} =& (4^{+5}_{-4})\times10^{-3}
      \end{align*}
    \end{column}
  \end{columns}
  \vspace{0.2cm}
  \begin{center}
    LHCb systematics not included yet, but central value of $\gamma$ remains high...\\
    ... it seems that the large tension with the LHCb global result $\gamma = (63.8^{+3.5}_{-3.7})^\circ$ remains
  \end{center}
\end{frame}

\begin{frame}{Interpretation results}
  \begin{center}
    {\large Results from interpretation of $h^+h^-\pi^+\pi^-$, after correcting for biases in central values (not uncertainties):}
  \end{center}
  \vspace{-0.5cm}
  \begin{columns}
    \begin{column}{0.5\textwidth}
      \begin{center}
        $K^+K^-\pi^+\pi^-$
      \end{center}
      \begin{align*}
        \gamma =& (117 \pm 15)^\circ \\
        \delta_B^{DK} =& (83 \pm 12)^\circ \\
        r_B^{DK} =& (12.1 \pm 2.6)\times10^{-2} \\
        \delta_B^{D\pi} =& (295 \pm 74)^\circ \\
        r_B^{D\pi} =& (0 \pm 5)\times10^{-3}
      \end{align*}
    \end{column}
    \begin{column}{0.5\textwidth}
      \begin{center}
        $\pi^+\pi^-\pi^+\pi^-$
      \end{center}
      \begin{align*}
        \gamma =& (45 \pm 11)^\circ \\
        \delta_B^{DK} =& (115 \pm 11)^\circ \\
        r_B^{DK} =& (8.2 \pm 1.9)\times10^{-2} \\
        \delta_B^{D\pi} =& (204 \pm 42)^\circ \\
        r_B^{D\pi} =& (4 \pm 5)\times10^{-3}
      \end{align*}
    \end{column}
  \end{columns}
  \vspace{0.3cm}
  \begin{center}
    $\pi^+\pi^-\pi^+\pi^-$ is in much better agreement with LHCb global result, but there is a tension with $K^+K^-\pi^+\pi^-$...\\
    \phantom{...but how Gaussian are these uncertainties?}
  \end{center}
\end{frame}

\begin{frame}{Interpretation results}
  \begin{center}
    {\large Results from interpretation of $h^+h^-\pi^+\pi^-$, after correcting for biases in central values (not uncertainties):}
  \end{center}
  \vspace{-0.5cm}
  \begin{columns}
    \begin{column}{0.5\textwidth}
      \begin{center}
        $K^+K^-\pi^+\pi^-$
      \end{center}
      \begin{align*}
        \gamma =& (117 \pm 15)^\circ \\
        \delta_B^{DK} =& (83 \pm 12)^\circ \\
        r_B^{DK} =& (12.1 \pm 2.6)\times10^{-2} \\
        \delta_B^{D\pi} =& (295 \pm 74)^\circ \\
        r_B^{D\pi} =& (0 \pm 5)\times10^{-3}
      \end{align*}
    \end{column}
    \begin{column}{0.5\textwidth}
      \begin{center}
        $\pi^+\pi^-\pi^+\pi^-$
      \end{center}
      \begin{align*}
        \gamma =& (45 \pm 11)^\circ \\
        \delta_B^{DK} =& (115 \pm 11)^\circ \\
        r_B^{DK} =& (8.2 \pm 1.9)\times10^{-2} \\
        \delta_B^{D\pi} =& (204 \pm 42)^\circ \\
        r_B^{D\pi} =& (4 \pm 5)\times10^{-3}
      \end{align*}
    \end{column}
  \end{columns}
  \vspace{0.3cm}
  \begin{center}
    $\pi^+\pi^-\pi^+\pi^-$ is in much better agreement with LHCb global result, but there is a tension with $K^+K^-\pi^+\pi^-$...\\
    ...but how Gaussian are these uncertainties?
  \end{center}
\end{frame}

\begin{frame}{Likelihood scan of interpretation fit}
  \begin{center}
    In fact, a likelihood scan shows that $D\to K^+K^-\pi^+\pi^-$ and $D\to\pi^+\pi^-\pi^+\pi^-$ agree within $2\sigma$ (no LHCb systematics yet)
  \end{center}
  \begin{figure}
    \centering
    \begin{subfigure}{0.50\textwidth}
      \centering
      \includegraphics[width=1.0\textwidth]{Plots/gamma_deltaB_hhpipi_LHCb_Prob_scan.pdf}
      \caption*{$\gamma$ vs $\delta_B^{DK}$}
    \end{subfigure}%
    \begin{subfigure}{0.50\textwidth}
      \centering
      \includegraphics[width=1.0\textwidth]{Plots/rB_deltaB_hhpipi_LHCb_Prob_scan.pdf}
      \caption*{$r_B^{DK}$ vs $\delta_B^{DK}$}
    \end{subfigure}
  \end{figure}
  \vspace{-0.3cm}
  \begin{center}
    When all biases, correlations and non-Gaussian uncertainties are accounted for, the tension with the LHCb average has reduced significantly
  \end{center}
\end{frame}

\begin{frame}{Likelihood scan of interpretation fit}
  \begin{center}
    In fact, a likelihood scan shows that $D\to K^+K^-\pi^+\pi^-$ and $D\to\pi^+\pi^-\pi^+\pi^-$ agree within $2\sigma$ (no LHCb systematics yet)
  \end{center}
  \begin{figure}
    \centering
    \begin{subfigure}{0.50\textwidth}
      \centering
      \includegraphics[width=1.0\textwidth]{Plots/gamma_deltaB_hhpipi_LHCb_Prob_scan.pdf}
      \caption*{$\gamma$ vs $\delta_B^{DK}$}
    \end{subfigure}%
    \begin{subfigure}{0.50\textwidth}
      \centering
      \includegraphics[width=1.0\textwidth]{Plots/rB_deltaB_hhpipi_LHCb_Prob_scan.pdf}
      \caption*{$r_B^{DK}$ vs $\delta_B^{DK}$}
    \end{subfigure}
  \end{figure}
  \vspace{-0.3cm}
  \begin{center}
    However, with all the non-Gaussian behaviour, are we sure these contours cover $68\%$ and $95\%$\phantom{y}?
  \end{center}
\end{frame}

\begin{frame}{Plugin/Feldman-Cousins method}
  \begin{center}
    {\Large Feldman-Cousins method, or Plugin, is a ``brute-force'' approach to assigning a confidence interval} \\~\\
    {At each scan point of $\gamma$, perform these fits to data:}
  \end{center}
  \begin{enumerate}
    \setlength\itemsep{0.5em}
    \item{Fit with all parameters floating, and save the log-likelihood $\chi^2$}
    \item{Fit with $\gamma$ fixed to scan point, and save $\chi^2_{\rm fix}$}
    \item{Calculate $\Delta\chi^2_{\rm data} = \chi^2_{\rm fix} - \chi^2$}
  \end{enumerate}
  \vspace{0.5cm}
  \begin{center}
    {We expect $\Delta\chi^2_{\rm data}$ to become large as we move away from best-fit value, but without direct knowledge of underlying PDF, we cannot determine any confidence intervals from this}
  \end{center}
\end{frame}

\begin{frame}{Plugin/Feldman-Cousins method}
  \begin{center}
    {\Large Feldman-Cousins method, or Plugin, is a ``brute-force'' approach to assigning a confidence interval} \\~\\
    {At each scan point of $\gamma$, perform these fits to toy:}
  \end{center}
  \begin{enumerate}
    \setlength\itemsep{0.5em}
    \item{Fix $\gamma$ to scan point and generate $1000$ toys}
    \item{Perform fits to each toy, with $\gamma$ both floating and fixed}
    \item{Calculate $\Delta\chi^2_{\rm toy}$}
  \end{enumerate}
  \vspace{0.42cm}
  \begin{center}
    {At each scan point, the fraction of toys with $\Delta\chi^2_{\rm toy} > \Delta\chi^2_{\rm data}$ is equal to $1 - \rm{CL}$, and the exact $68\%$ confidence interval can then be obtained using an interpolation between points}
  \end{center}
\end{frame}

\begin{frame}{Plugin/Feldman-Cousins method}
  \begin{center}
    LHCb average within $2\sigma$ of $D\to K^+K^-\pi^+\pi^-$ Plugin result \\
    Combined fit shows good agreement between Plugin and Prob scans
  \end{center}
  \begin{figure}
    \centering
    \includegraphics[width=0.6\textwidth]{Plots/gamma_plugin_scan.pdf}
  \end{figure}
  \vspace{-0.3cm}
  \begin{center}
    Combined fit result: $\gamma = (57 \pm 9)^\circ$\\
    Third most precise single measurement of $\gamma$ in $B^\pm$ decays
  \end{center}
\end{frame}

\section{Summary and next steps}

\begin{frame}{Summary and next steps}
  \vspace{0.0cm}
  {\Large In summary:}
  \vspace{0.5cm}
  \begin{enumerate}
    \setlength\itemsep{1.0em}
    \item{BESIII results approved, paper in review}
    \item{Model-independent measurement of $\gamma$ with $B^\pm\to[h^+h^-\pi^+\pi^-]_Dh^\pm$ presented to B2OC, and ANA note circulated to B2OC conveners}
    \item{$3\sigma$ tension in $D\to K^+K^-\pi^+\pi^-$ has reduced to less than $2\sigma$ due to:}
    \begin{enumerate}
      \item{Non-Gaussian uncertainties in $y_\pm^{DK}$ originating from $s_i$ uncertainties}
      \item{Large anti-correlation between $\gamma$ and $\delta_B^{DK}$}
    \end{enumerate}
    \item{Main takeaway: Important to meausure $\gamma$ \underline{model independently}!}
  \end{enumerate}
\end{frame}

\begin{frame}{Summary and next steps}
  \vspace{0.0cm}
  {\Large Next steps:}
  \vspace{0.5cm}
  \begin{itemize}
    \setlength\itemsep{1.0em}
    \item{Aim to finish thesis by the end of June}
    \item{Unfortunately I failed to obtain useful TORCH results...}
    \item{...but it was a useful experience with testbeam and RICH work!}
    \item{Future plans: Start new postdoc position in Heidelberg in September}
  \end{itemize}
  \vspace{0.4cm}
  \begin{center}
    {\huge Thanks for your attention and thanks for all your support during my PhD!}
  \end{center}
\end{frame}

\end{document}